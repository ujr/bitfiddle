% The Science of the Bit Fiddle
% Assumes the dvipdfm processor
% ujr/2020-12-13

\input epsf
\input fonts

\ifx\pdfoutput\undefined\csname newcount\endcsname\pdfoutput\fi
\ifcase\pdfoutput\def\pdfinfo#1{\special{pdf:docinfo<<#1>>}}\fi
\pdfinfo{ /Author (Urs-Jakob R^^fcetschi)
 /Title (The Science of the Bit Fiddle)
 /Subject (Analysis of Rob Miles's Bitshift Variations in C Minor) }
% Overly simplitic hyperlinks...
\def\wurl#1#2{\special{html:<a href="#1">}#2\special{html:</a>}}%

\parindent=0pt
\parskip=0pt
\vsize=25.5 truecm
\hsize=13.0 truecm
\advance\hoffset by .5truecm
\font\tf=cmssbx12 scaled\magstep2 % title font
\font\bs=cmssbx10 % bold sans
\nopagenumbers
\def\newline{\hfil\break}\let\\=\newline
\catcode`\|\active\def|{\kern3pt}%

\font\tenss=cmss10
\font\eightss=cmss8
\def\mseven#1{{\tenss#1m\raise.5ex\hbox{\eightss 7}}}% minor seventh

\newbox\sidebox
\setbox\sidebox=\vtop{\hsize=3cm%\smallskip
\halign{\hskip3pt\hfil\rm#&\quad#&\hfil\tt#\cr
\omit\hfil\it\ dec\hfil&&\omit\hfil\it binary\hfil\cr
 0&& 0000|0000\cr
 1&& 0000|0001\cr
 2&& 0000|0010\cr
 3&& 0000|0011\cr
 4&& 0000|0100\cr
 5&& 0000|0101\cr
 6&& 0000|0110\cr
 7&& 0000|0111\cr
\noalign{\smallskip}
 8&& 0000|1000\cr
 9&& 0000|1001\cr
10&& 0000|1010\cr
11&& 0000|1011\cr
12&& 0000|1100\cr
13&& 0000|1101\cr
14&& 0000|1110\cr
15&& 0000|1111\cr
\noalign{\smallskip}
16&& 0001|0000\cr
17&& 0001|0001\cr
18&& 0001|0010\cr
19&& 0001|0011\cr
20&& 0001|0100\cr
21&& 0001|0101\cr
22&& 0001|0110\cr
23&& 0001|0111\cr
\noalign{\smallskip}
24&& 0001|1000\cr
25&& 0001|1001\cr
26&& 0001|1010\cr
27&& 0001|1011\cr
28&& 0001|1100\cr
29&& 0001|1101\cr
30&& 0001|1110\cr
31&& 0001|1111\cr
%\noalign{\smallskip}
%32&& 0010|0000\cr
%33&& 0010|0001\cr
%34&& 0010|0010\cr
%35&& 0010|0011\cr
%36&& 0010|0100\cr
%37&& 0010|0101\cr
%38&& 0010|0110\cr
%39&& 0010|0111\cr
%\noalign{\smallskip}
%40&& 0010|1000\cr
%41&& 0010|1001\cr
%42&& 0010|1010\cr
%43&& 0010|1011\cr
%44&& 0010|1100\cr
%45&& 0010|1101\cr
%46&& 0010|1110\cr
%47&& 0010|1111\cr
%\noalign{\smallskip}
%48&& 0011|0000\cr
%49&& 0011|0001\cr
%50&& 0011|0010\cr
%51&& 0011|0011\cr
\noalign{\medskip}
 63&& 0011|1111\cr
127&& 0111|1111\cr
255&& 1111|1111\cr
}
\bigskip\hrule
\bigskip\hsize=2.6cm
\centerline{\it Timing Divisors}
\centerline{\it at 8192\kern.5ex Hz}
\smallskip
{\tt i>>10} $\equiv$ {\tt i}/1024\newline
\centerline{8 times per sec}
\smallskip
{\tt i>>11} $\equiv$ {\tt i}/2048\newline
\centerline{4 times per sec}
\smallskip
{\tt i>>12} $\equiv$ {\tt i}/4096\newline
\centerline{2 times per sec}
\smallskip
{\tt i>>13} $\equiv$ {\tt i}/8192\newline
\centerline{once per sec}
\smallskip
{\tt i>>14} $\equiv$ {\tt i}/16384\newline
\centerline{two seconds}
\smallskip
{\tt i>>17} $\equiv$ {\tt i}/\rlap{131072}\newline
\centerline{16 seconds}
\vskip7pt\hrule
\medskip
\centerline{U.\kern.5ex J.\kern.5ex R\"uetschi}
\line{\hss 2020 CC\kern.5ex BY\kern-.3ex-SA\hss}
\vskip4pt\hrule\vskip3pt
\centerline{\wurl{http://github.com/ujr/bitfiddle}{more on GitHub}}
}

\leftline{\tf The Science of the Bit Fiddle%
\vadjust{\vbox to 0pt{\moveright\hsize\rlap{\hskip6pt
 \vrule\hskip3pt\box\sidebox}\vss}}}

\bigskip\noindent
Notes and tables to accompany an analysis of\newline
{\bs Rob Miles}'s 2016 {\bs Bitshift Variations in C Minor}
\smallskip\noindent
Original code: http://txti.es/bitshiftvariationsincminor\newline
Intro movie: https://www.youtube.com/watch?v=MqZgoNRERY8\newline
Recording: https://soundcloud.com/robertskmiles/bitshift-variations-in-c-minor

\medskip
\line{\hrulefill\bs\quad Basics\quad\hrulefill}
\halign{#\hfil&\quad\hfil\tt#&\qquad#\hfil&\quad\hfil\tt#&\qquad#\hfil\cr
Masking& i|=|xxxx|xxxx&Shifting&i=22|=|0001|0110&shifting {\tt<<} and {\tt>>}\cr
       &\llap{mask}|=|0001|1111&&i<<2|=|0101|1000&binds tighter than\cr
       &\llap{mask}\&i|=|000x|xxxx&&i>>2|=|0000|0101&masking {\tt\&}\cr}

\medskip
\line{\hrulefill\bs\quad Sawtooth Sound\quad\hrulefill}
\medskip
\line{\hfil\epsfbox{bitfiddle-1.mps}\hfil
 \epsfbox{bitfiddle-2.mps}\hfil\epsfbox{bitfiddle-3.mps}\hfil}
\medskip
\line{\hfil\epsfbox{bitfiddle-4.mps}\hfil
 \epsfbox{bitfiddle-5.mps}\hfil\epsfbox{bitfiddle-6.mps}\hfil}
\medskip
\centerline{\epsfbox{bitfiddle-7.mps}} % sound generator
\centerline{Sound generator:
 $i$ = clock, $n$ = pitch indicator, $o$ = octave, $4$ = volume}

\medskip
\line{\hrulefill\bs\quad Just Intonation\quad\hrulefill}
\halign to\hsize{\tabskip=1pc plus 1ex minus .5ex
  \hfil#&\hfil#\hfil&\hfil#\hfil&\hfil#\hfil&\hfil#\hfil&\hfil
  #\hfil&\hfil#\hfil&\hfil#\hfil&#\hfil\cr
 char:&\tt'\%'&\tt'6'&\tt'B'&\tt'Q'&\tt'Y'&\tt'j'&\tt'\char"7D'\cr
ASCII:& 37&  54&  66&  81&  89& 106& 125\cr
  +51:& 88& 105& 117& 132& 140& 157& 176\cr
ratio:& $1$ &$1.19$&$1.32$&$1.50$&$1.59$&$1.78$&$2$& the ratios of\cr
      &$1/1$&$6/5$&$4/3$&$3/2$&$8/5$&$16/9$&$2/1$&
   \llap{\smash{\raise1.3ex\hbox{$\Bigr\}$}} }just intonation\cr
 note:& C & E$\flat$ & F& G& A$\flat$& B$\flat$& C$'$& \hfil C minor\cr}

\smallskip
\line{\hrulefill\bs\quad Melodies\quad\hrulefill}
\smallskip\noindent
Assume playback at 8192 Hz (close to the actual 8000 Hz
but gives nice figures).
The expression {\tt 3\&(i>>16)} yields the repeating sequence
0 1 2 3 and progresses every $2^{16}$ increments of $i$ or
about every 8 seconds. It chooses between two sets of notes:
\smallskip
\halign{\tabskip=2ex\quad\hfil#&\tt#\hfil&#\hfil&\hfil#\hfil&#\hfil\cr
1.&BY\char"7D\relax6YB6\%&
 or \ F A$\flat$ C$'$ E$\flat$ A$\flat$ F E$\flat$ C&
 (\mseven F)&for {\tt 3\&(i>>16)} in $1,2,3$\cr
2.&Qj\char"7D\relax6jQ6\%&
 or \ G B$\flat$ C$'$ E$\flat$ B$\flat$ G E$\flat$ C&
 (\mseven C)&for {\tt 3\&(i>>16)} $=0$\cr}
\smallskip
\noindent{\bf Voice 1:} controlled by ${\tt n}={\tt i}/2^{14}$,
which increments every 2 seconds. The current set of notes
is indexed by {\tt n\%8}, so we get the 1st half of set~2,
the 2nd half of set~1, then the entire set~1, before the
melody repeats. Period is thus 32 seconds.
\vskip7mm
\line{\special{pdf:epdf width 13.5cm (voice1.pdf)}\hss}
\vskip-2mm
\noindent{\bf Voices 2, 3, 4:} left as an exercise...

\smallskip
\line{\hrulefill\bs\quad Outlook\quad\hrulefill}
\smallskip\noindent
Omitted here: remaining voices, voice variation, harmonies,
period length (it turns out to be 960 seconds).
Terseness: exploiting operator precedence, type defaults ({\tt int}),
arguments to main are really just local variables defined using
as little code as possible.
How was it invented? Certainly by genius!
But remember the law of downhill synthesis and uphill analysis
(V.~Braitenberg in his 1986 book {\sl Vehicles\/}).
I~found these bitshift variations
a {\it fun\/} analysis of a {\it fascinating\/} synthesis.
\smallskip\hrule

\bye
